\documentclass[10pt,a4paper]{article}
\usepackage[utf8]{inputenc}
\usepackage[russian]{babel}
\usepackage[OT1]{fontenc}
\usepackage{amsmath}
\usepackage{amsfonts}
\usepackage{amssymb}
\usepackage{graphicx}
\author{Рикардо санчес}
\date{}
\title{Отчет по лабораторной работе №4 Nmap + Metasploit}
\begin{document}
\maketitle

\newpage

\section{NMAP}

\subsection{Цель работы}

Научиться пользоваться утлитой nmap.

\subsection{Ход работы}

\subsubsection{Подготовка}

Скачаны дистрибутивы Kali linux и Metasploitable2, развернуты на virtualbox, тип сетевого поделючения - сетевой мост.

\subsubsection{Поиск активных хостов}

Поиск активных хостов путем ICMP ping. Данный способ может не сработать в реальных условиях, т.к. в большинстве корпоротивных сетей блокируется из соображений безопасности.

\begin{verbatim}
root@debian:~# nmap --version

Nmap version 6.47 ( http://nmap.org )
Platform: x86_64-unknown-linux-gnu
Compiled with: nmap-liblua-5.2.3 openssl-1.0.1e libpcre-8.30 libpcap-1.3.0 nmap-libdnet-1.12 ipv6
Compiled without:
Available nsock engines: epoll poll select
root@debian:~# nmap -sn 192.168.0.*

Starting Nmap 6.47 ( http://nmap.org ) at 2015-06-11 01:15 EDT
Nmap scan report for 192.168.0.1
Host is up (0.0079s latency).
MAC Address: E8:94:F6:FA:47:16 (Tp-link Technologies Co.)
Nmap scan report for 192.168.0.2
Host is up (0.054s latency).
MAC Address: B8:C6:8E:A2:E3:38 (Samsung Electronics Co.)
Nmap scan report for 192.168.0.150
Host is up (0.0021s latency).
MAC Address: 30:10:B3:0C:BB:B4 (Liteon Technology)
Nmap scan report for 192.168.0.151
Host is up (0.00083s latency).
MAC Address: 08:00:27:FF:A2:B1 (Cadmus Computer Systems)
Nmap scan report for 192.168.0.152
Host is up (0.0060s latency).
MAC Address: B8:C6:8E:A2:E3:38 (Samsung Electronics Co.)
Nmap scan report for 192.168.0.153
Host is up (0.024s latency).
MAC Address: C0:D9:62:7D:50:25 (Askey Computer)
Nmap scan report for 192.168.0.154
Host is up (0.0022s latency).
MAC Address: 08:00:27:1A:CA:E6 (Cadmus Computer Systems)
Nmap scan report for 192.168.0.156
Host is up (0.0010s latency).
MAC Address: 08:00:27:D6:48:F3 (Cadmus Computer Systems)
Nmap scan report for 192.168.0.155
Host is up.
Nmap done: 256 IP addresses (9 hosts up) scanned in 3.65 seconds
root@debian:~# 

\end{verbatim}

итого id Metasploit able 2

\begin{itemize}
\item 192.168.0.156  Metasploit able 2
\end{itemize}

Полученые результаты соответствуют действительности.

\subsubsection{Поиск открытых портов}

Для этих целей будем использовать уязвиную машину Metaspoitable 2.

\begin{verbatim}

root@debian:~# nmap 192.168.0.156

Starting Nmap 6.47 ( http://nmap.org ) at 2015-06-11 01:21 EDT
Nmap scan report for 192.168.0.156
Host is up (0.0011s latency).
Not shown: 976 closed ports
PORT      STATE SERVICE
21/tcp    open  ftp
22/tcp    open  ssh
23/tcp    open  telnet
25/tcp    open  smtp
53/tcp    open  domain
80/tcp    open  http
111/tcp   open  rpcbind
139/tcp   open  netbios-ssn
445/tcp   open  microsoft-ds
512/tcp   open  exec
513/tcp   open  login
514/tcp   open  shell
1099/tcp  open  rmiregistry
1524/tcp  open  ingreslock
2049/tcp  open  nfs
2121/tcp  open  ccproxy-ftp
3306/tcp  open  mysql
5432/tcp  open  postgresql
5900/tcp  open  vnc
6000/tcp  open  X11
6667/tcp  open  irc
8009/tcp  open  ajp13
8180/tcp  open  unknown
50003/tcp open  unknown
MAC Address: 08:00:27:D6:48:F3 (Cadmus Computer Systems)

Nmap done: 1 IP address (1 host up) scanned in 0.66 seconds
root@debian:~# 

\end{verbatim}

\subsubsection{Определение версии сервисов}

\begin{verbatim}

root@debian:~# nmap 192.168.0.156 -sV

Starting Nmap 6.47 ( http://nmap.org ) at 2015-06-11 01:27 EDT
Nmap scan report for 192.168.0.156
Host is up (0.00059s latency).
Not shown: 976 closed ports
PORT      STATE SERVICE     VERSION
21/tcp    open  ftp         vsftpd 2.3.4
22/tcp    open  ssh         OpenSSH 4.7p1 Debian 8ubuntu1 (protocol 2.0)
23/tcp    open  telnet      Linux telnetd
25/tcp    open  smtp        Postfix smtpd
53/tcp    open  domain      ISC BIND 9.4.2
80/tcp    open  http        Apache httpd 2.2.8 ((Ubuntu) DAV/2)
111/tcp   open  rpcbind     2 (RPC #100000)
139/tcp   open  netbios-ssn Samba smbd 3.X (workgroup: WORKGROUP)
445/tcp   open  netbios-ssn Samba smbd 3.X (workgroup: WORKGROUP)
512/tcp   open  exec        netkit-rsh rexecd
513/tcp   open  login?
514/tcp   open  tcpwrapped
1099/tcp  open  rmiregistry GNU Classpath grmiregistry
1524/tcp  open  shell       Metasploitable root shell
2049/tcp  open  nfs         2-4 (RPC #100003)
2121/tcp  open  ftp         ProFTPD 1.3.1
3306/tcp  open  mysql       MySQL 5.0.51a-3ubuntu5
5432/tcp  open  postgresql  PostgreSQL DB 8.3.0 - 8.3.7
5900/tcp  open  vnc         VNC (protocol 3.3)
6000/tcp  open  X11         (access denied)
6667/tcp  open  irc         Unreal ircd
8009/tcp  open  ajp13       Apache Jserv (Protocol v1.3)
8180/tcp  open  http        Apache Tomcat/Coyote JSP engine 1.1
50003/tcp open  unknown
MAC Address: 08:00:27:D6:48:F3 (Cadmus Computer Systems)
Service Info: Hosts:  metasploitable.localdomain, localhost, irc.Metasploitable.LAN; OSs: Unix, Linux; CPE: cpe:/o:linux:linux_kernel

Service detection performed. Please report any incorrect results at http://nmap.org/submit/ .
Nmap done: 1 IP address (1 host up) scanned in 167.96 seconds
root@debian:~# 

\end{verbatim}

\subsubsection{Сохраняем вывод утлиты в Xml}

\begin{verbatim}

root@debian:~# nmap 192.168.0.156 -sV -oX /home/nmapOutput.txt
Starting Nmap 6.47 ( http://nmap.org ) at 2015-06-11 01:33 EDT
Nmap scan report for 192.168.0.156
Host is up (0.00072s latency).
Not shown: 976 closed ports
PORT      STATE SERVICE     VERSION
21/tcp    open  ftp         vsftpd 2.3.4
22/tcp    open  ssh         OpenSSH 4.7p1 Debian 8ubuntu1 (protocol 2.0)
23/tcp    open  telnet      Linux telnetd
25/tcp    open  smtp        Postfix smtpd
53/tcp    open  domain      ISC BIND 9.4.2
80/tcp    open  http        Apache httpd 2.2.8 ((Ubuntu) DAV/2)
111/tcp   open  rpcbind     2 (RPC #100000)
139/tcp   open  netbios-ssn Samba smbd 3.X (workgroup: WORKGROUP)
445/tcp   open  netbios-ssn Samba smbd 3.X (workgroup: WORKGROUP)
512/tcp   open  exec        netkit-rsh rexecd
513/tcp   open  login
514/tcp   open  tcpwrapped
1099/tcp  open  rmiregistry GNU Classpath grmiregistry
1524/tcp  open  shell       Metasploitable root shell
2049/tcp  open  nfs         2-4 (RPC #100003)
2121/tcp  open  ftp         ProFTPD 1.3.1
3306/tcp  open  mysql       MySQL 5.0.51a-3ubuntu5
5432/tcp  open  postgresql  PostgreSQL DB 8.3.0 - 8.3.7
5900/tcp  open  vnc         VNC (protocol 3.3)
6000/tcp  open  X11         (access denied)
6667/tcp  open  irc         Unreal ircd
8009/tcp  open  ajp13       Apache Jserv (Protocol v1.3)
8180/tcp  open  http        Apache Tomcat/Coyote JSP engine 1.1
50003/tcp open  unknown
MAC Address: 08:00:27:D6:48:F3 (Cadmus Computer Systems)
Service Info: Hosts:  metasploitable.localdomain, localhost, irc.Metasploitable.LAN; OSs: Unix, Linux; CPE: cpe:/o:linux:linux_kernel

Service detection performed. Please report any incorrect results at http://nmap.org/submit/ .
Nmap done: 1 IP address (1 host up) scanned in 167.45 seconds
root@debian:~# 

\end{verbatim}

Полученный файл находится в репозитории.

\subsubsection{Изучить файлы nmap-services, nmap-os-db, nmap-service-probes}

Данные файлы находятся в репозитории.


\begin{itemize}
\item {nmap-services}
Представляет собой таблицу в которой содержатся информация о сервисах, типу и номеру порта, и частоте появления.

\item {nmap-os-db}
Содержит информацию о сигнатурах рахличных ОС. 

Пример записи:

\begin{verbatim}
# 2-Wire 2701HG-G Gateway Software:  5.29.133.27
Fingerprint 2Wire 2701HG-G wireless ADSL modem
Class 2Wire | embedded || WAP
CPE cpe:/h:2wire:2701hg-g
SEQ(SP=7B-85%GCD=1-6%ISR=95-9F%TI=I%II=I%SS=S%TS=A)
OPS(O1=M5B4NNSW0NNNT11%O2=M578NNSW0NNNT11%O3=M280W0NNNT11%O4=M218NNSW0NNNT11%O5=M218NNSW0NNNT11%O6=M109NNSNNT11)
WIN(W1=8000%W2=8000%W3=8000%W4=8000%W5=8000%W6=8000)
ECN(R=Y%DF=Y%T=FA-104%TG=FF%W=8000%O=M5B4NNSW0N%CC=N%Q=)
T1(R=Y%DF=Y%T=FA-104%TG=FF%S=O%A=S+%F=AS%RD=0%Q=)
T2(R=N)
T3(R=Y%DF=Y%T=FA-104%TG=FF%W=8000%S=O%A=S+%F=AS%O=M109NNSW0NNNT11%RD=0%Q=)
T4(R=N)
T5(R=Y%DF=Y%T=FA-104%TG=FF%W=0%S=Z%A=S+%F=AR%O=%RD=BD1AB510%Q=)
T6(R=N)
T7(R=N)
U1(DF=Y%T=FA-104%TG=FF%IPL=70%UN=0%RIPL=G%RID=G%RIPCK=G%RUCK=G%RUD=G)
IE(DFI=Y%T=FA-104%TG=FF%CD=S)
\end{verbatim}

\item {nmap-service-probes}
Содержит скрипт для определения сервиса, запущенного на данном порте. 

Пример записи:

\begin{verbatim}
Probe TCP NessusTPv10 q|< NTP/1.0 >\n|
rarity 8
ports 1241
sslports 1241

match http-proxy m|^HTTP/1\.0 400 Bad Request\r\nServer: squid/([\w._+-]+)\r\n| p/Squid/ v/$1/ cpe:/a:squid-cache:squid:$1/

match nessus m|^< NTP/1.0 >\n| p/Nessus Daemon/ i/NTP v1.0/ cpe:/a:tenable:nessus/
match zabbix m|^NOT OK\n$| p/Zabbix Monitoring System/ cpe:/a:zabbix:zabbix/

\end{verbatim}


\end{itemize}

\subsubsection{Добавление своей сигнатуры}

В качестве сервера была использована утлита netcat:

\begin{verbatim}
root@kali:~# (echo -e "GoodMorningpolitex\nVersion 4.3.2.1";) | nc -vv -l -p 5000
\end{verbatim}

Сигнатура: 

\begin{verbatim}
Probe TCP SimpleServer q|Any text|

match simple tcp m|GoodMorningpolitex\nVersion ([0-9.]*)|
p/Simple Server/ v/$P(1)/
\end{verbatim}

Из ответа извлекается версия и возвращается в качестве ответа.

\begin{verbatim}

root@debian:~# nmap 192.168.0.155 -p 5000 -sV

Starting Nmap 6.47 ( http://nmap.org ) at 2015-06-11 02:48 EDT
Nmap scan report for 192.168.0.155
Host is up (0.00034s latency).
PORT     STATE  SERVICE VERSION
5000/tcp closed upnp

Service detection performed. Please report any incorrect results at http://nmap.org/submit/ .
Nmap done: 1 IP address (1 host up) scanned in 0.73 seconds
root@debian:~# 

\end{verbatim}

\subsubsection{Исследование различных способов сканирования с помощью утлиты WireShark}

Исследовались следующие варианты сканирования портов для машины-цели с адресом 192.168.0.153

Wireshark является программой-анализатором сетевых пакетов с исходным кодом. Без какого-либо специального оборудования или перенастройки эта программа может перехватывать входящие и исходящие данные на любом сетевом интерфейсе компьютера: Ethernet, WiFi, PPP, loopback и даже USB. Обычно Wireshark применяется для выявления проблем в сети, таких, как перегруженность, слишком долгое время ожидания или ошибки протоколов. Но для того, чтобы изучить Wireshark, совсем не нужно ждать, когда произойдет какая-либо поломка. Давайте приступим к обзору этой программы.


Перехват трафика


Запуск новой сессии перехвата производится в окне программы из меню "Capture". Чтобы увидеть весь список сетевых интерфейсов, которые смогла обнаружить Wireshark, перейдите по пути в меню "Capture > Interfaces". Появится диалоговое окно, в котором, помимо физических устройств, будет присутствовать псевдо-устройство "any", которое перехватывает данные со всех других устройств этого списка.
Перед началом можно задать некоторые опции, с которыми будет запускаться перехват. Перейдя по "Capture > Options", достаточно выбрать:

- фильтры для выборочного анализа трафика (например, по определенному протоколу или диапазону адресов);
- автоматически остановить перехват по достижении указанного в настройках времени;
- отсортировать полученные данные по указанному размеру или дате.

Первое, что вы увидите при запуске новой сессии - окно лога, где будет показываться основная информация о выполняемом программой процессе: источник, приемник, протокол, время и т.п. Вся информация организована в виде таблицы с заголовками. Для большей удобочитаемости Wireshark выполняет цветовое выделение фрагментов текста, изменение цвета фона или пометку наиболее "интересных" пакетов с помощью флагов.


\subsection{Выводы}

 работы были изучены основные возможности nmap. Определение активных хостов, сканирование портов, определение версий сервисов, дополнение определения версий сервисов, были рассмотрены основные файлы используемые для определения версий сервисов и ОС. 
\end{document}