\documentclass[12pt,a4paper]{article}
\usepackage[utf8]{inputenc}
\usepackage[russian]{babel}
\usepackage{amsmath}
\usepackage{amsfonts}
\usepackage{amssymb}
\usepackage{graphicx}
\usepackage{listings}

\author{Рикардо санчес}
\date{}
\title{Отчет по лабораторной работе №5 SSL}
\begin{document}
\maketitle
\section{Цель работы}
Научиться развертывать SSL/TLS сервер.
\section{Ход работы}

\subsection{Лучшие практики по развертыванию SSL/TLS}
1. Использовать 2048-битные ключи.\\
2. Закрывайте приватные ключи.\\
3. Позабодьтесь о достаточном покрытии доменных имен.\\
4. Получайте сертификаты у надежных CA.\\
5. Используйте криптостойкие алгоритмы для подписей.\\
6. Настраивайте сервер для работы с несколькими сертификатами.\\
7. Используйте безопасные протоколы.\\
8. Используйте криптостойкие алгоритмы шифрования. Ключ не меннее 128 бит.\\
9. Используйте Forward Secrecy, позволяющую защищенный обмен, не зависящий от приватного ключа сервера.\\
10. Если нет необходимости, отключайте проверку защищенности соединения на стороне клиента.\\
11. Адаптируйте свою систему. Устанавливайте патчи к модулям защиты, когда они появляются.\\
12. Надо найти комромис между защищенностью системы и производительностью.\\
13. Шифруйте 100\% вашего сайта.\\
14. Используйте защищенные куки.\\

\subsection{Изучить основные уязвимости и атаки на SSL последнего времени – POODLE, HeartBleed}

\paragraph{HeartBleed}
Уязвимости подвержаны следующие версии
OpenSSL 1.0.1 до 1.0.1f включительно.

Суть ошибки - неконтролируемое переполнение буфера, позволяющее несанкционированно читать память на сервере или на клиенте, в том числе для извлечения закрытого ключа сервера. Информация об уязвимости была опубликована в апреле 2014 года, ошибка существовала с конца 2011 года.


\paragraph{POODLE}
Суть уязвимости: злоумышленник может заставить обе стороны перейти на ssl 3.0, в котором используется потоковое шифрование RC4, которое, при больших объемах трафка, позволяет получить информацию, помогающее дешифрованию.


\subsection{Практическое задание: Выбрать со стартовой страницы SSL Server Test один домен из списка Recent Best и один домен из списка Recent Worst – изучить отчеты, интерпретировать результаты в разделе Summary}

\subsubsection{SSL Report: bankorange.ru (84.52.66.47)}
\begin{figure}[h!]
\centering
\includegraphics[scale=0.5]{2}
\caption{SSL Report: bankorange.ru}
\end{figure}
Этот сервер уязвимым для POODLE attack, использует слабую подпись. Поддерживает только устаревшие протоколы. Сервер принимает уязвимое RC4 шифрование. Так же сервер не поддерживает Forward Secrecy.
\subsubsection{SSL Report: luxexpress.eu (217.146.69.16)}
\begin{figure}[h!]
\centering
\includegraphics[scale=0.5]{1}
\caption{SSL Report: luxexpress.eu}
\end{figure}
У сервера нет доверенного сертификата.Сертификат имеет слабую подпись и истекает после 2016. Использует алгоритм Диффи-Хеллмана для обмена ключами. Имеет неполную цепочку сертификатов. Но защищен от DownGrade атаки.
\small
\begin{lstlisting}
TLS_ECDHE_RSA_WITH_AES_256_GCM_SHA384 (0xc030)   ECDH 256 bits (eq. 3072 bits RSA)   FS	256
TLS_ECDHE_RSA_WITH_AES_128_GCM_SHA256 (0xc02f)   ECDH 256 bits (eq. 3072 bits RSA)   FS	128
TLS_ECDHE_RSA_WITH_AES_256_CBC_SHA384 (0xc028)   ECDH 256 bits (eq. 3072 bits RSA)   FS	256
TLS_ECDHE_RSA_WITH_AES_128_CBC_SHA256 (0xc027)   ECDH 256 bits (eq. 3072 bits RSA)   FS	128
TLS_ECDHE_RSA_WITH_AES_256_CBC_SHA (0xc014)   ECDH 256 bits (eq. 3072 bits RSA)   FS	256
TLS_ECDHE_RSA_WITH_AES_128_CBC_SHA (0xc013)   ECDH 256 bits (eq. 3072 bits RSA)   FS	128
\end{lstlisting}
\subsection{Прокомментировать большинство позиций в разделе Protocol Details}
\begin{verbatim}
Protocol Details
Secure Renegotiation	Supported
Secure Client-Initiated Renegotiation	No
Insecure Client-Initiated Renegotiation	No
BEAST attack	Not mitigated server-side (more info)   TLS 1.0: 0x35
POODLE (SSLv3)	No, SSL 3 not supported (more info)
POODLE (TLS)	No (more info)
Downgrade attack prevention	Yes, TLS_FALLBACK_SCSV supported (more info)
TLS compression	No
RC4	No
Heartbeat (extension)	No
Heartbleed (vulnerability)	No (more info)
OpenSSL CCS vuln. (CVE-2014-0224)	No (more info)
Forward Secrecy	No   WEAK (more info)
Next Protocol Negotiation (NPN)	No
Session resumption (caching)	Yes
Session resumption (tickets)	No
OCSP stapling	No
Strict Transport Security (HSTS)	Disabled   max-age=0
Public Key Pinning (HPKP)	No
Long handshake intolerance	No
TLS extension intolerance	No
TLS version intolerance	TLS 1.98 	TLS 2.98 
Incorrect SNI alerts	-
Uses common DH prime	No
SSL 2 handshake compatibility	No
\end{verbatim}
\begin{itemize}
\item Secure Renegotiation - Хранение доп параметров о TLS соединении.
\item BEAST attack - защита от BEAST атаки.
\item POODLE (SSLv3) - защита от пуделя по SSLv3.
\item POODLE (TLS) - защита от пуделя по TLS.
\item Downgrade attack prevention - защита от downgrade атаки.
\item TLS compression - сжатие данных по tls.
\item RC4 - использование алгоритма шифрования RC4
\item Heartbleed - защита от уязвимости Heartbleed.
\item OpenSSL CCS vuln. - SSL ChangeCipherSpec уязвимость.
\item Forward Secrecy - защищенный обмен без приватного ключа сервера.
\item Next Protocol Negotiation - расширение SSL, позволяющее договариваться о протоколе соединения.
\item OCSP - Online Certificate Status Protocol проверка валидности сертификата.
\item Strict Transport Security (HSTS) - механизм, активирующий форсированное защищённое соединение по HTTPS.
\item Public Key Pinning (HPKP) - фиксирует привязку публичного ключа к даннолму узлу.
\item Long handshake intolerance - поддержка длинных(больще 256 байт) handshake сообщений.
\item TLS extension intolerance - поддержка TLS расширений.
\item Incorrect SNI alerts - предупреждение при некорректном Server Name Indication.
\end{itemize}
\subsection{Сделать итоговый вывод о реализации SSL на заданном домене}
На домене bankorange.ru (84.52.66.47) отсутствует защита от POODLE attack, имеется слабая электронная подпись, отсутствует поддержка Forward Secrecy.

\section{Выводы}

В ходе данной работы были изучены "best practice" использования SSL/TLS. Были рассмотрены основные возможности сервиса Qualys SSL Labs – SSL Server Test. Данный сервис позволяет провести анализ качества защищенности домена. В качестве резюме можно получить статус самых известных уязвимостей для данной сервера, а также информацию о поддерживаемых протоколах и режимах работы. Кроме того, сервис тут же предлагает дополнительную информацию по вопросам решения указанных проблем. 

\end{document}
\end{document}

